\documentclass[a4paper,tikz]{article}

\usepackage[frenchb]{babel}
\usepackage[utf8]{inputenc}
\usepackage[T1]{fontenc}
\usepackage{amsmath}
\usepackage{graphicx}
\usepackage[colorinlistoftodos]{todonotes}
\usepackage{a4wide}
\usepackage{enumitem}

\usepackage[usenames,dvipsnames]{xcolor}
\definecolor{dkgreen}{rgb}{0,0.6,0}
\definecolor{steelblue}{rgb}{0.16,0.37,0.58}
\definecolor{gray}{rgb}{0.5,0.5,0.5}
\definecolor{mauve}{rgb}{0.58,0,0.82}
\definecolor{blue}{rgb}{0,0,0.7}
\definecolor{hlColor}{rgb}{0.94,0.94,0.94}
\definecolor{shadecolor}{rgb}{0.96,0.96,0.96}
\definecolor{TFFrameColor}{rgb}{0.96,0.96,0.96}
\definecolor{TFTitleColor}{rgb}{0.00,0.00,0.00}
\definecolor{lightred}{rgb}{1,0.96,0.96}
\definecolor{darkred}{rgb}{0.85,0.33,0.31}
\definecolor{lightblue}{HTML}{EBF5FA}
\definecolor{lightblue2}{HTML}{E3F2FA}
\definecolor{darkblue}{HTML}{D2DCE1}
\definecolor{lightyellow}{HTML}{FFFAE6}
\definecolor{darkyellow}{HTML}{FAE6BE}

\usepackage{listings}
\lstset{
	language=C,
	basicstyle=\scriptsize,
	numbers=left,                   % where to put the line-numbers
  	numberstyle=\tiny\color{gray},
	commentstyle=\color{steelblue},
	stringstyle=\color{BrickRed},
	backgroundcolor=\color{shadecolor},
    keywordstyle=\color{OliveGreen},
	frame=single,                   % adds a frame around the code
 	rulecolor=\color{black},
	emph={},
	emphstyle=\color{mauve},
	morekeywords=[2]{},
	keywordstyle=[2]{\color{dkgreen}},
	showstringspaces=false,
  	tabsize=4,
	moredelim=[is][\small\ttfamily]{/!}{!/},
	breaklines=true
}

\usepackage{hyperref}
\hypersetup{
	colorlinks=true, % false: boxed links; true: colored links
	linkcolor=black, % color of internal links
	urlcolor=blue,   % color of external links
	citecolor=blue
}
\newcommand{\hhref}[1]{\href{#1}{#1}}

\usepackage{makecell}

\usepackage{eurosym} %\euro -> €

\title{INF5101A - TP1}

\date{\today}

\begin{document}
\maketitle
\newpage
\tableofcontents
\newpage

\section{Distribution des données}

\begin{lstlisting}
for k = 0 to n-2
	for i = (k+1) to n-1
		ratio = -a[i][k] / a[k][k]
		for j = (k+1) to n-1
			a[i][j] += ratio * a[k][j]
\end{lstlisting}
\

Pour un $k$ donné, nous pouvons paralléliser la boucle $i$ (ou $j$). Une première
idée pour paralléliser serait de diviser la matrice en $N/Nt$ ($N$ taille de la
matrice, $Nt$ nombre de tâches). La première tâche s'occuperait ainsi des $N/Nt$
premières lignes, et ainsi de suite pour chaque tâche. Cependant, procéder ainsi
impliquerait que les premières tâches aient moins de travail que les dernières.
En effet, si nous prenons par exemple $N = 16$, et $Nt = 4$, alors la première
tâche n'aurait que 3 itérations de boucles de $k$ à effectuer (0 à 2), alors que
la dernière tâche en aurait 15 à effectuer (0 à 14). Répartir de manière cyclique
les lignes sur les différentes tâches serait ainsi plus efficace.

\section{Chargement en parallèle de la matrice}

\lstinputlisting[firstline=13, lastline=48]{../gaussp.c}

\lstinputlisting[firstline=50, lastline=59]{../gaussp.c}
\newpage

\section{Gauss parallèle}

\lstinputlisting[firstline=120, lastline=157]{../gaussp.c}
\newpage

\section{Sauvegarde en parallèle de la matrice}

\lstinputlisting[firstline=73, lastline=107]{../gaussp.c}

\lstinputlisting[firstline=61, lastline=71]{../gaussp.c}
\newpage

\section{Relevés}

\begin{figure}[h!]
	\centering
	\caption{Computation time}
	\begin{tabular}{|l|l|l|l|l|}
		\hline
		\diaghead{taille matriceeee}{Taille matrice}{Nb. tâches} & 1 & 4 & 10 & 20 \\ \hline
		500 & 0.111 s & 1.140 s & 1.390 s & 1.265 s \\ \hline
		1000 & 0.855 s & 4.169 s & 4.240 s & 4.308 s \\ \hline
		5000 & 112.046 s & 126.909 s & 100.502 s & 97.863 s \\ \hline
		10000 & 856.741 s & 640.369 s & 467.9 s & 400 s \\ \hline
	\end{tabular}
\end{figure}

\begin{figure}[h!]
	\centering
	\caption{Speedup}
	\begin{tabular}{|l|l|l|l|l|}
		\hline
		\diaghead{taille matriceeee}{Taille matrice}{Nb. tâches} & 1 & 4 & 10 & 20 \\ \hline
		500 && 0.097368421 & 0.079856115 & 0.087747036 \\ \hline
		1000 && 0.205085152 & 0.201650943 & 0.198467967 \\ \hline
		5000 && 0.882884587 & 1.114863386 & 1.144927092 \\ \hline
		10000 && 1.337886437 & 1.831034409 & 2.1418525 \\ \hline
	\end{tabular}
\end{figure}

\begin{figure}[h!]
	\centering
	\caption{Efficacité}
	\begin{tabular}{|l|l|l|l|l|}
		\hline
		\diaghead{taille matriceeee}{Taille matrice}{Nb. tâches} & 1 & 4 & 10 & 20 \\ \hline
		500 && 0.024342105 & 0.007985612 & 0.004387352 \\ \hline
		1000 && 0.051271288 & 0.020165094 & 0.009923398 \\ \hline
		5000 && 0.220721147 & 0.111486339 & 0.057246355 \\ \hline
		10000 && 0.334471609 & 0.183103441 & 0.107092625 \\ \hline
	\end{tabular}
\end{figure}
\newpage

\section{Listing complet}

\lstinputlisting{../gaussp.c}

\end{document}
